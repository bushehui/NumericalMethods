\subsection{对称正定矩阵的平方根法和$LDL^T$分解法}

\frame{
当$A$是对称正定矩阵时,存在一个实的非奇异下三角矩阵$L_1$,使
\begin{equation*}
A = L_1 L_1^T
\end{equation*}
当限定$L_1$的对角线为正时,这种分解唯一。
\begin{center}
$\Downarrow$
\end{center}
这种分解称为矩阵的Cholesky分解。
}

\frame{
\begin{block}{定理}
设$A$是对称正定矩阵,则$A$有如下分解:
\begin{equation*}
A = L D L^T
\end{equation*}
其中,$L$是单位下三角阵,$D$为对角阵,且这种分解是唯一的。
\end{block}
Proof: \\
因为$A$对称正定,从而$A$有唯一的LU分解
\begin{equation*}
A = 
\left[ 
\begin{array}{c c c c}
1 & & & \\
l_{21} & 1 & & \\
\vdots & \vdots & \ddots & \\
l_{n1} & l_{n2} & \cdots & 1
\end{array}
\right] 
\left[ 
\begin{array}{c c c c}
u_{11} & u_{12} & \cdots & u_{1n} \\
 & u_{22} & \cdots & u_{2n} \\
 &  & \ddots & \vdots \\
 &  &  & u_{nn} \\
\end{array}
\right] 
\end{equation*}
\begin{center}
$\Downarrow$
\end{center}
}

\frame{
因为:
$\Delta_k = u_{11} \cdots u_{kk} \neq 0$ ($k = 1, 2, \cdots, n$),故上式可化为
\begin{block}{}
$A = $
\begin{equation*}
\left[ 
\begin{array}{c c c c}
1 & & & \\
l_{21} & 1 & & \\
\vdots & \vdots & \ddots & \\
l_{n1} & l_{n2} & \cdots & 1
\end{array}
\right] 
\left[ 
\begin{array}{c c c c}
u_{11} &  &  &  \\
 & u_{22} &  &  \\
 &  & \ddots &  \\
 &  &  & u_{nn} \\
\end{array}
\right] 
\left[ 
\begin{array}{c c c c}
1 & u_{12} \slash u_{11}  & \cdots & u_{1n}\slash u_{11} \\
 & 1 & \cdots & u_{2n} \slash u_{22} \\
 &  & \ddots & \vdots \\
 &  &  & 1 \\
\end{array}
\right] 
\end{equation*}
\end{block}
\begin{center}
$\Downarrow$
\end{center}
将右端三角矩阵分别记为$L$,$D$和$R$。\\
因为$A$对称,有 \\
\begin{center}
$A = L D R = R^T D L^T$
或者
$A = L D R = R^T ( D L^T )$
\end{center}
\begin{center}
$\Downarrow$
\end{center}
因为$A$的$LU$分解唯一,故$L = R^T$,$L^T = R$,$A = LDL^T$,此分解显然是唯一的。
}

\frame{
\begin{block}{定理}
$n$阶对称正定矩阵$A$一定有Cholesky分解$A = L_1 L_1^T$ $\Rightarrow$ 当限定$L_1$的对角线为正时,矩阵的Cholesky分解唯一。
\end{block}
证明:\\
由上一个定理可知:$D = diag ( d_1, d_2, \cdots, d_n )$ $\Rightarrow$
使
$A = L D^{1 \slash 2} (L D^{1 \slash 2})^T$ \\
\begin{center}
$\Downarrow$
\end{center}
又记$L_1 = L D^{1 \slash 2}$,为一个非奇异下三角阵,一对角线元素为正,故$A$有Cholesky分解
$A = L_1 L_1^T$
\begin{center}
$\Downarrow$
\end{center}
分解的唯一性显而易见。
}

\frame{
\frametitle{Cholesky分解的计算公式}
设 $A = L L^T$,即
\begin{equation*}
\left[ 
\begin{array}{c c c c}
a_{11} & a_{12} & \cdots & a_{1n} \\
a_{21} & a_{22} & \cdots & a_{2n} \\
\vdots & \vdots & \ddots & \vdots \\
a_{n1} & a_{n2} & \cdots & a_{nn} 
\end{array}
\right] 
=
\left[ 
\begin{array}{c c c c}
l_{11} &  &  &  \\
l_{21} & l_{22} &  &  \\
\vdots & \vdots & \ddots &  \\
l_{n1} & l_{n2} & \cdots & l_{nn} 
\end{array}
\right] 
\left[ 
\begin{array}{c c c c}
l_{11} & l_{12} & \cdots & l_{1n} \\
 & l_{22} & \cdots & l_{2n} \\
 &  & \ddots & \vdots \\
 &  &  & l_{nn} 
\end{array}
\right] 
\end{equation*}
其中$a_{ij} = a_{ji}$,($i,j = 1, 2, \cdots, n$);
$l_{ii} > 0$,($i,j = 1, 2, \cdots, n$) \\ 
\vspace{2mm}
第一步:\\
由矩阵乘法有$a_{11} = l_{11}^2$,且$a_{i1} = l_{l_i1} l_{11}$,故求得 \\
$l_{11} = \sqrt{a_{11}}$,且$l_{i1} = a_{i1} \slash l_{11}$,($i,j = 1, 2, \cdots, n$) \\
\begin{center}
$\Downarrow$
\end{center}
}

\frame{
设$L$矩阵的前$k-1$列元素已经求出,\\
则第$k$步:
由矩阵乘法可得
\begin{equation*}
\sum_{m=1}^{k-1} l_{km}^2 + l_{kk}^2 = a_{kk} 
\hspace{10mm} 
\sum_{m=1}^{k-1} l_{im} l_{km} + l_{ik} l_{kk} = a_{ik} 
\end{equation*}
\begin{center}
$\Downarrow$
\end{center}
\begin{equation*}
l_{kk} = \sqrt{ a_{kk} - \sum_{m=1}^{k-1} l_{km}^2}  
\end{equation*}
\begin{equation*}
 l_{ik} = \frac{ a_{ik} - \sum_{m=1}^{k-1} l_{im} l_{km}} { l_{kk}} 
\end{equation*}
$i = k+1, \cdots, n$;$k = 1, 2, \cdots, n$ 
\begin{center}
$\Downarrow$
\end{center}
由于分解公式中的每一步都有开方运算,故又称Cholesky方法为平方根法。
}


\frame{
由于
\begin{equation*}
a_{kk} = \sum_{m=1}^{k-1} l_{km}^2
\end{equation*}
\begin{center}
$\Downarrow$
\end{center}
\begin{equation*}
 l_{km}^2 \le a_{kk} \le \max_{1 \le k \le n} a_{kk}
\end{equation*}
消元法的舍入误差的放大受到限制,$\Rightarrow$ Cholesky方法不用考虑选主元的问题。
}

\frame{
\begin{block}{Cholesky方法的算法}
\begin{itemize}
\item 输入$A$,$b$,$\varepsilon$
\item $k=1, 2, \cdots, n$ :
\begin{itemize}
\item 计算$S_k = a_{kk} - \sum_{m=1}^{k-1} l^2_{km}$
\item 若$S_k < \varepsilon $ ,则打印“求解失败”,出错处理
\item 否则计算$l_{kk}=\sqrt{a_{kk} - \sum_{m=1}^{k-1} l_{km}^2}$,$l_{ik} = \frac{ a_{ik} - \sum_{m=1}^{k-1} l_{im} l_{km}} { l_{kk}} $
\end{itemize}
\item 求解$LY = b$
\begin{itemize}
\item $y_1 = b_1 \slash l_{11}$
\item 对$i = 2, 3, \cdots, n$,计算 $y_i = (b_i - \sum_{j=1}^{i-1} l_{ij} y_j ) \slash l_{ii}$
\end{itemize}
\item 求解$L^T X = Y$
\begin{itemize}
\item $x_n = y_n \slash l_{nn}$
\item 对$i = n-1, n-2, \cdots, 1$,计算$x_i = (y_i - \sum_{j= i+1}^n l_{ji} x_j ) \slash l_{ii} $
\end{itemize}
\item 打印$x_i$
\end{itemize}
\end{block}
}



%\subsection{向量与矩阵范数}

%\frame{
%}
