\subsection{Upper-Triangular Linear Systems}


\frame{
\begin{itemize}
\item We will now develop the {\Large  back-substitution algorithm},  which is useful for solving a linear system of equations that has an upper-triangular coefficient matrix. 
\item This algorithm will be incorporated in the algorithm for solving a general linear system.
\end{itemize}
\vspace{3mm}
\begin{block}{Definition.}
\begin{itemize}
\item An  $N  \times N$ matrix  $A = [ a_{i , j} ]$ is called {\Large upper triangular} provided that the elements satisfy $ a_{i , j} = 0 $ whenever $ i > j $ . 
\item The $N \times N$ matrix  $A = [a_{i , j} ]$ is called {\Large lower triangular} provided that $a_{i,j} = 0$ whenever $i < j$.
\end{itemize}
\end{block}
}

\frame{
If $A$ is an upper-triangular matrix, then $A X = B$ is said to be an {\Large upper triangular system } of linear equations and has the form
\begin{equation*}
\begin{array}{ rl }
a_{1,1} x_1 + a_{1,2} x_2  + a_{1,3} x_3 + \cdots +  a_{1, N-1} x_{N-1} + a_{1, N} x_N & = b_1 \\
a_{2,2} x_2  + a_{2,3} x_3 + \cdots +  a_{2, N-1} x_{N-1} + a_{2, N} x_N & = b_2 \\
a_{3,3} x_3 + \cdots +  a_{3, N-1} x_{N-1} + a_{3, N} x_N & = b_3 \\
\vdots \ \ \ \ \ \ \ \ \ \ \ \ \ \ \ \ & \ \ \vdots \\
a_{N-1, N-1} x_{N-1} + a_{N-1, N} x_N & = b_{N-1} \\
a_{N,N} x_N & = b_N 
\end{array}
\end{equation*}
\begin{block}{Theorem (Back Substitution).}
Suppose that $A X = B$ is an upper-triangular system with the form given in above upper triangular system. 
If  $a_{k, k} \ne 0 \ \ \ for \ \ k = 1, 2, \ldots, N$,
then there exists a {\Large unique solution} to this upper triangular system.
\end{block}
}

\frame{
\begin{block}{Constructive Proof of Theorem:} 
\begin{itemize}
\item The last equation involves only $x_N$, so we solve it first:
\begin{equation*}
x_N = \frac{b_N}{a_{N,N}} 
\end{equation*}
\begin{center}
$\Downarrow$
\end{center}
\item Now $x_N$ is known and it can be used in the next-to-last equation:
\begin{equation*}
x_{N-1} = \frac{b_{N-1} - a_{N-1,N}x_N}{a_{N-1,N-1}}
\end{equation*}
\begin{center}
$\Downarrow$
\end{center}
\item Now $x_N$ and $x_{N-1}$ are used to find $x_{N-2}$ :
\begin{equation*}
x_{N-2} = \frac{b_{N-2} - a_{N-2,N-1}x_{N-1} - a_{N-2,N}x_N }{a_{N-2,N-2}}
\end{equation*}
\end{itemize}
\end{block}
}

\frame{
\begin{block}{Constructive Proof of Theorem (continued):} 
\vspace{5mm}
\begin{center}
$\Downarrow$
\end{center}
\begin{itemize}
\item Once the values $x_N, x_{N-1}, \ldots, x_{k+1}$ are known, the general step is
\begin{equation*}
x_k = \frac{b_k - \sum_{j = k + 1}^N  a_{k, j}x_j }{a_{k, k}}
\end{equation*}
\begin{center}
$\Downarrow$
\end{center}
\item The uniqueness of the solution is easy to see. 
\begin{itemize}
\item The $N$-th equation implies that $b_N  \slash a_{N, N}$ is the {\Large only possible value} of $x_N$ .
\item Then finite induction is used to establish that $x_{N-1}, x_{N-2}, \cdots , x_1$ are {\Large unique}.
\end{itemize}
\end{itemize}
\end{block}
}


\frame{
\frametitle{Example .}
Use {\Large back substitution} to solve the linear system
\begin{equation*}
\begin{array}{r l}
4x_1 - x_2 + 2x_3 + 3x_4 = & 20 \\
-2x_2 + 7x_3 - 4x_4 = & -7 \\
6x_3 + 5x_4 = & 4 \\
3x_4 = & 6.
\end{array}
\end{equation*}
\begin{center}
$\Downarrow$
\end{center}
Solving for $x_4$ in the last equation yields 
\begin{equation*}
x_4 = \frac{6}{3} = 2 
\end{equation*}
\begin{center}
$\Downarrow$
\end{center}
Using $x_4 = 2$ in the third equation, we obtain
\begin{equation*}
x_3 = \frac{4-5(2)}{6} = -1 
\end{equation*}
}

\frame{
\begin{center}
$\Downarrow$
\end{center}
Now $x_3 = -1$ and $x_4 = 2$ are used to find $x_2$ in the second equation:
\begin{equation*}
x_2 = \frac{-7 - 7 (-1) + 4(2)}{-2} = -4 
\end{equation*}
\begin{center}
$\Downarrow$
\end{center}
Finally, $x_1$ is obtained using the first equation:
\begin{equation*}
x_1 = \frac{20 + 1 (-4)  - 2 (-1) - 3(2)}{4} = 3 
\end{equation*}
\begin{block}{}
\begin{itemize}
\item The condition that $a_{k,k}  \ne 0$ is essential. 
\item If this requirement is not fulfilled, either no solution exists or infinitely many solutions exist.
\end{itemize}
\end{block}
}

\frame{
\frametitle{Example.}
Show that there is no solution to the linear system
\begin{equation*}
\begin{array}{r l}
4x_1 -   x_2 + 2x_3 + 3x_4 = & 20 \\
0x_2 + 7x_3 - 4x_4 = & -7 \\
6x_3 + 5x_4 = & 4 \\
3x_4 = & 6.
\end{array}
\end{equation*}
\begin{center}
$\Downarrow$
\end{center}
Using the last equation, we must have $x_4 = 2$, 
which is substituted into the second and third equations to obtain
\begin{equation*}
\begin{array}{r c r c r c r r}
7x_3 & - & 8   & = & -7 & \rightarrow & x_3 = & 1/7 \\
6x_3 & + & 10 & = & 4   & \rightarrow & x_3 = & -1
\end{array}
\end{equation*}
\begin{center}
$\Downarrow$
\end{center}
\begin{block}{}
This contradiction leads to the conclusion that there is no solution to the linear system .
\end{block}
}

\frame{
\frametitle{Example}
Show that there are infinitely many solutions to
\begin{equation*}
\begin{array}{r l}
4x_1 - x_2 + 2x_3 + 3x_4 = & 20 \\
0x_2 + 7x_3 + 0x_4 = & -7 \\
6x_3 + 5x_4 = & 4 \\
3x_4 = & 6.
\end{array}
\end{equation*}
\begin{center}
$\Downarrow$
\end{center}
Using the last equation, we must have $x_4 = 2$,
\begin{center}
$\Downarrow$
\end{center}
Substituted the $x_4 = 2$ into the second and third equations to get $x_3 = -1$.
\begin{center}
$\Downarrow$
\end{center}
}

\frame{
\begin{center}
$\Downarrow$
\end{center}
Only two values $x_3$ and $x_4$ have been obtained, and when they are substituted into the first equation, the result is
\begin{equation*}
x_2 = 4 x_1 - 16
\end{equation*}
\begin{center}
$\Downarrow$
\end{center}
If we choose a value of $x_1$, then the value of $x_2$ is uniquely determined.
\begin{center}
$\Downarrow$
\end{center}
\begin{block}{}
The values of $x_1$ and $x_2$ can not be uniquely determined by this linear system.
\end{block}
}



\frame{
\begin{block}{Theorem.}
If the $ N \times N $ matrix $ A = [ a_{i,j} ] $ is either upper or lower triangular, then
\begin{equation*}
det (A) =a_{1,1}  a_{2,2} \cdots a_{N,N} = \prod_{i=1}^N a_{i, i}
\end{equation*}
\end{block}
\begin{center}
$\Downarrow$
\end{center}
\begin{itemize}
\item The linear system $A X = B$, where $A$ is an $N \times N$ matrix,  has a unique solution if and only if $ det (A) \ne 0$.
\item If any entry on the main diagonal of an upper- or lower-triangular matrix is zero,  then $det(A) = 0$.
\end{itemize}
\begin{center}
$\Downarrow$
\end{center}
The proof  can be found in most introductory linear algebra textbooks.
}


