\setcounter{section}{6}



\section{矩阵特征值和特征向量的数值解法}

\frame{
\frametitle{特征值 $\&$ 特征向量}
设$A$是$n \times n$矩阵,
\begin{center}
$\Downarrow$
\end{center}
如果存在数$\lambda$和$n$维非零向量$x$,
满足$A x = \lambda x$,
\begin{center}
$\Downarrow$
\end{center}
则称$\lambda$为矩阵$A$的一个特征值,$x$称为与$\lambda$对应的特征向量。
}



\subsection{幂法}

\frame{
设$A$具有$n$个线性无关的特征向量$x_1$, $x_2$, $\cdots$, $x_n$,
其相应的特征值为$\lambda_1$, $\lambda_2$, $\cdots$, $\lambda_n$满足
\begin{equation*}
|\lambda_1| \ge |\lambda_2| \ge |\lambda_3| \ge \cdots \ge |\lambda_n|
\end{equation*}
现任取一个非零向量$\mu_0$,令
\begin{equation*}
\mu_k = A \mu_{k-1}  \hspace{5mm} (k = 1, 2, \cdots )
\end{equation*}
得向量序列$\{ \mu_k \}$,$k = 0, 1, 2, \cdots$
}

\frame{
因为$x_1$, $x_2$, $\cdots$, $x_n$线性无关,
故$n$维向量$\mu_0$可由它们线性表示为
\begin{equation*}
\mu_0 = a_1 x_1 + a_2 x_2 + \cdots + a_n x_n
\end{equation*}
所以
\begin{equation*}
\begin{array}{l l}
\mu_k & = A \mu_{k-1}  = A^2 \mu_{k-2} = \cdots =   A^k \mu_{0} \\
& \\
& = a_1 A^k x_1 + a_2 A^k x_2 + \cdots + a_n A^k x_n  \\
& \\
& = a_1 \lambda_1^k x_1 + a_2 \lambda_2^k x_2 + \cdots + a_n \lambda_n^k x_n  \\
& \\
& = \lambda_1^k \left[ a_1 x_1 + a_2 \left( \frac{\lambda_2}{\lambda_1} \right)^k x_2 + \cdots + a_n \left( \frac{\lambda_n}{\lambda_1} \right)^k x_n   \right]\\
\end{array}
\end{equation*}
设$a_1 \neq 0$,因为$\left| \frac{\lambda_2}{\lambda_1} \right| < 1$,$\cdots$,
当 $k$充分大时,有$\mu_k \approx \lambda_1^k a_1 x_1$,且为非零向量,
所以$\mu_k$可以近似的作为$\lambda_1$对应的特征向量。
}

\frame{
为防止$\mu_k$的模过大或者过小产生的计算机运算的上下溢出,
通常每次迭代都对$\mu_k$进行归一化,
使$||\mu_k||_{\infty} = 1$
\begin{center}
$\Downarrow$
\end{center}
幂法公式改进为
\begin{equation*}
y_{k-1} = \frac{\mu_{k-1}}{||\mu_{k-1}||_{\infty} }
\end{equation*}
\begin{equation*}
\mu_k = A y_{k-1} \hspace{5mm} (k = 1, 2, \cdots )
\end{equation*}
\begin{center}
$\Downarrow$
\end{center}
由迭代公式知
\begin{equation*}
\mu_k = \frac{A \mu_{k-1}}{||\mu_{k-1}||_{\infty} } = \frac{ \frac{A^2 \mu_{k-2}}{||\mu_{k-2}||_{\infty}} }{\left| \left| \frac{A \mu_{k-2}}{||\mu_{k-2}||_{\infty} } \right| \right|_{\infty} } 
= \frac{A^2 \mu_{k-2}}{||A \mu_{k-2}||_{\infty}}
= \cdots =
\frac{A^k \mu_0}{||A^{k-1} \mu_0||_{\infty}}
\end{equation*}
}

\frame{
当$k$相当大时,$\mu_k$和$\mu_{k-1}$都可以近似为$\lambda_1$对应的特征向量
\begin{equation*}
\mu_k = A \frac{A \mu_{k-1}}{||\mu_{k-1}||_{\infty} }
\end{equation*}
有
\begin{equation*}
\mu_k = \lambda_1 \frac{\mu_{k-1}}{||\mu_{k-1}||_{\infty}}
\end{equation*}
\begin{equation*}
||\mu_k||_{\infty} = |\lambda_1|  \frac{||\mu_{k-1}||_{\infty}}{||\mu_{k-1}||_{\infty}} =  |\lambda_1| 
\end{equation*}
\begin{center}
$\Downarrow$
\end{center}
故$\lambda_1$就是$||\mu_{k}||_{\infty}$剩下的问题是确定$\lambda_1$的符号。
$||\mu_{k}||_{\infty}$是$\mu_k$的各分量中绝对值最大者
}

\frame{
设$\mu_k$和$\mu_{k-1}$的绝对值最大的分量为$a_k$和$a_{k-1}$,
因此可得
\begin{equation*}
a_k = \lambda_1 \frac{a_{k-1}}{||\mu_{k}||_{\infty}}
\end{equation*}
\begin{center}
$\Downarrow$
\end{center}
\begin{equation*}
\lambda_1 = \frac{a_k}{a_{k-1}} ||\mu_{k-1}||_{\infty}
\end{equation*}
\begin{equation*}
sgn(\lambda_1) = sgn \left( \frac{a_k}{a_{k-1}} \right)
\end{equation*}
\begin{block}{}
经过迭代之后,$ ||\mu_{k}||_{\infty}$稳定后,取$|\lambda_1| =  ||\mu_{k}||_{\infty}$,
而$\lambda_1$的符号由以下原则确定:
\begin{itemize}
\item 当相邻两次的$\mu_k$和$\mu_{k-1}$对应分量符号相同,$\lambda_1$的符号为正;
\item 否则为负。
\end{itemize}
\end{block}
}

\frame{
\begin{block}{幂法算法}
给定$n \times n$矩阵$A$,精度为$\varepsilon$,任取非零向量$\mu_0$,对$k = 1, 2, \cdots$,
执行以下各步骤:
\vspace{3mm}
\begin{itemize}
\item $\textcircled{1}$ 设$|a_r| = \max_{1 \le i \le n} |a_i|$,其中$a_i$是$\mu_{k-1} = (a_1, a_2, \cdots, a_n)^T$的各个分量
\vspace{3mm}
\item $\textcircled{2}$ $y_{k-1} = \mu_{k-1} \slash  |a_r|$
\vspace{3mm}
\item $\textcircled{3}$ $\mu_k = A y_{k-1}$
\vspace{3mm}
\item $\textcircled{4}$ $t_k = sgn(a_r) \times (\mu_k)_r$
\vspace{3mm}
\item $\textcircled{5}$ 若$|t_k - t_{k-1}| < \varepsilon$,令$\lambda_1 = t_k$,$x_1 = y_{k-1}$,推出运算;否则返回$\textcircled{1}$
\end{itemize}
\end{block}
}

%\frame{
%\frametitle{幂法的几点问题}
%\begin{itemize}
%\item 当$A$不具有$n$个线性无关的特征向量时,幂法不适用。$\Rightarrow$当使用幂法运算时,如果发现不收敛,要考虑此种可能。
%\item 

%\end{itemize}
%}



\subsection{反幂法}

\frame{
由于$A x_i = \lambda_i x_i$,可以推出
\begin{equation*}
A^{-1} x_i = \frac{1}{\lambda_i} x_i
\end{equation*}
\begin{center}
$\Downarrow$
\end{center}
若有$|\lambda_1| \ge |\lambda_2| \ge \cdots |\lambda_{n-1}| \ge |\lambda_n|$,
则$\frac{1}{\lambda_n}$是$A^{-1}$的按模最大的特征值
\begin{center}
$\Downarrow$
\end{center}
用幂法求出$A^{-1}$的按模最大特征值,即求出$A$的按模最小的特征值。
\begin{center}
$\Downarrow$\\
\end{center}
\begin{center}
这种方法称为:{\huge 反幂法}
\end{center}
}

%\frame{
%反幂法的迭代公式变为$\mu_k = A^{-1} y_{k-1}$ $\Rightarrow$ 要先计算逆阵$A^{-1}$。
%\begin{center}
%$\Downarrow$\\
%\end{center}
%为避免求逆矩阵,需要
%}


\frame{
\begin{block}{反幂算法}
\begin{itemize}
\item $\textcircled{1}$ 对$A$进行LU分解,求出L,U
\item $\textcircled{2}$ 对任意非零向量$\mu_0$,$k=1, 2, \cdots$执行
\begin{itemize}
\item $a)$ $|a_r| = \max_{1 \le i \le n} |a_i|$,$a_i$是$\mu+{k-1} = (a_1, a_2, \cdots, a_n)^T$的分量
\item $b)$ 
\begin{equation*}
y_{k-1} \frac{\mu_{k-1}}{|a_r|}
\end{equation*}
\item $c)$ 解
\begin{equation*}
\left\{ 
\begin{array}{l}
L x_{k-1} = y_{k-1} \\
U \mu_k = x_{k-1}
\end{array}
\right.
\end{equation*}
得到$\mu_k$
\item $d)$ 
\begin{equation*}
t_k = \frac{sgn(a_r)}{(\mu_k)_r}
\end{equation*}
\item $e)$ 当$|t_{k} - t_{k-1}| < \varepsilon$, 令$\lambda_n = t_k$,$x_n = y_{k-1}$,退出运算;否则返回$a)$
\end{itemize}
\end{itemize}
\end{block}
}

\frame{
\frametitle{原点平移加速}
设矩阵$B = A - tI$,$t$为选择平移量,$I$为单位矩阵。
\begin{center}
$\Downarrow$
\end{center}
$A$的特征值为$\lambda_1$, $\lambda_2$, $\cdots$, $\lambda_n$,
则$B$的特征值 $\lambda_1 - t$, $\lambda_2 - t$, $\cdots$, $\lambda_n - t$,
且$A$,$B$的特征向量相同
\begin{center}
$\Downarrow$
\end{center}
适当选择$t$,使$\lambda_1 - t$仍是$B$的按模最大特征值,且
\begin{equation*}
\left| \frac{\lambda_2 - t}{\lambda_1 - t} \right| < \left| \frac{\lambda_2}{\lambda_1} \right|
\end{equation*}
\begin{center}
$\Downarrow$
\end{center}
对$B$用幂法求$\lambda_1 - t$,比直接对$A$求$\lambda_1$要快。
$\Rightarrow$
求出$\lambda_1 - t$自然而然也就求出$\lambda_1$
}

\frame{
设$A$的特征值都为实数,且$\lambda_1 > \lambda_2 \ge \cdots \ge \lambda_{n-1} > \lambda_n$
\begin{center}
$\Downarrow$
\end{center}
不论如何选择$t$,$B$的按模最大特征值要么是$\lambda_1 -t$,要么$\lambda_n -t$
\begin{center}
$\Downarrow$
\end{center}
选择$t$应该满足
\begin{equation*}
|\lambda_1 - t| \ge |\lambda_n - t|
\end{equation*}
\begin{equation*}
\max \left\{ \frac{|\lambda_2 - t|}{|\lambda_1 - t|}, \frac{|\lambda_n - t|}{|\lambda_1 - t|} \right\} = \min
\end{equation*}
\begin{center}
$\Downarrow$
\end{center}
当$\lambda_2 - t = - ( \lambda_n - t) $,即$t = \frac{\lambda_2 + \lambda_n}{2}$时,收敛速度最快。
$\Rightarrow$
$\lambda_2$与$\lambda_n$事先是未知的,故上述结论只有理论的意义。
}


\frame{
\frametitle{求已知特征值的特征向量}
若已知$A$的某一特征值$\lambda_m$的近似值$\lambda'_m$
\begin{center}
$\Downarrow$
\end{center}
\begin{equation*}
|\lambda_m - \lambda'_m| = \min |\lambda_i - \lambda'_m|
\end{equation*}
按原点平移法的思想,取$t = \lambda'_m$ $B = A - \lambda'_m I$,
则$B$的按模最小的特征值是$\lambda_m - \lambda'_m$,
对应的特征向量则与$A$的对应$\lambda_m$的特征向量相同
\begin{center}
$\Downarrow$
\end{center}
对$B$做反幂法:
取初始向量$z_0$,做迭代
\begin{equation*}
\left\{
\begin{array}{l c l}
B y_k & = & z_{k-1} \\
m_k  & = & \max (y_k) \\
z_k   & = & y_k \slash m_k  \hspace{3mm} (k = 1, 2, \cdots) \\
\end{array} 
\right.
\end{equation*}
其中$\max(y_k)$表示$y_k$的绝对值最大的分量。
}

\frame{
当$B y_k = z_{k-1}$时,也可先对$B$作LU分解,以免重复分解工作。
\begin{center}
$\Downarrow$
\end{center}
\begin{equation*}
\lim_{k \to \infty} m_k = 
\end{equation*}
\begin{equation*}
\lim_{k \to \infty} z_k = 
\end{equation*}

}


\subsection{雅科比(Jacobi)方法}

\frame{
\begin{equation*}
1
\end{equation*}
}

\frame{
}



\subsection{QR算法}

\frame{
}


