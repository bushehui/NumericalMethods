\subsection{复化求积公式}



\frame{
\begin{itemize}
\item 当$n \to \infty$时,并非对所有的$f(x)$,都有
\begin{equation*}
\int_a^b f(x) dx - \sum_{i=0}^n (b-a) c_i^{(n)} f(x_i) \to 0
\end{equation*}
这说明Newton-Cotes求积公式的收敛性对某些被求积函数$f(x)$得不到保证。
\item 当$n \ge 8$时,Newton-Cotes求积公式的稳定性也得不到保证。
\end{itemize}
\begin{center}
$\Downarrow$
\end{center}
\begin{block}{}
为了提高积分近似公式的精度,可以采用复化公式
\begin{itemize}
\item  复化梯形求积公式
\item  复化Simpson求积公式
\end{itemize}
\end{block}
}

\frame{
\frametitle{复化梯形求积公式}
把积分区间$[a, b]$进行$n$等分,得到$n+1$个点$x_0$, $x_1$, $\cdots$, $x_k$, $\cdots$, $x_n$,
其中$x_k = a + k h$;$k = 0, 1, \cdots, n$;$h = (b-a) \slash n$, 
使用这些分点把区间$[a, b]$进行$n$个小区间$[x_0, x_1]$, $[x_1, x_2]$, $\cdots$, $[x_{n-1}, x_n]$。
根据定积分的性质和梯形求积公式可得:
\begin{equation*}
\begin{array}{l l l}
\int_a^b f(x) dx & = & \int_{x_0}^{x_1} f(x) dx + \int_{x_1}^{x_2} f(x) dx + \cdots + \int_{x_{n-1}}^{x_n} f(x) dx \\
& & \\
& = & \sum_{k=0}^{n-1} \int_{x_k}^{x_{k+1}} f(x) dx \\
& & \\
& \approx & \sum_{k=0}^{n-1} \frac{x_{k+1} - x_k}{2} [f(x_k) + f(x_{k+1})] \\
& & \\
& = & \frac{h}{2} \sum_{k=0}^{n-1} [f(x_k) + f(x_{k+1})] \\
& & \\
& = & \frac{h}{2}  [f(a) + f(b)  + 2 \sum_{k=1}^{n-1} f(a + kh) ]\\
\end{array}
\end{equation*}
\begin{center}
$\Downarrow$
\end{center}
记为$T_n$,其中$n$表示等分数。
}

\frame{
\frametitle{复化梯形求积公式的求积分算法}
\begin{block}{}
\begin{itemize}
\item 输入$a$,$b$和$n$,
\vspace{3mm}
\item $h = \frac{b-a}{n}$
\vspace{3mm}
\item Loop :
\begin{itemize}
\item $sum = 0$
\vspace{3mm}
\item 对 $k = 1, 2, \cdots, n-1$做
\vspace{3mm}
\item $sum = sum + f(a + k h)$
\end{itemize}
\vspace{3mm}
\item $T = \frac{h}{2} [f(a) + f(b) + 2 \ast sum]$
\vspace{3mm}
\item 输出$T$
\end{itemize}
\end{block}
}

\frame{
\begin{block}{定理}
若$f(x) \in C^2 [a, b]$, 则对复化梯形求积公式$T_n$有
\begin{equation*}
\int_a^b f(x) dx - T_n = - \frac{b-a}{12} h^2 f''(\eta)
\end{equation*}
其中$h = \frac{b-a}{n}$,$a \le \eta \le b$
\end{block}
定理证明:\\ 
把积分区间$[a, b]$进行$n$等分,得到分点$x_0$, $x_1$, $\cdots$, $x_k$, $\cdots$, $x_n$。
\begin{center}
$\Downarrow$
\end{center}
因此得到$n$个小区间$[x_0, x_1]$, $[x_1, x_2]$, $\cdots$, $[x_{n-1}, x_n]$。
}

\frame{
\begin{center}
$\Downarrow$
\end{center}
在这些小区间上直接使用梯形求积公式的误差估计式,可得:
\begin{equation*}
\begin{array}{l c l}
\int_a^b f(x) d x -T_n & = & \sum_{k=0}^{n-1} \left\{ \int_{x_k}^{x_{k+1}} f(x) dx -\frac{h}{2} \left[ f(x_k) + f(x_{k+1}) \right] \right\} \\
&  & \\
& = & -\frac{h^3}{12} \sum_{k=0}^{n-1} f''(\eta_k) \\
\end{array}
\end{equation*}
其中$x_k \le \eta_k \le x_{k+1}$
\begin{center}
$\Downarrow$
\end{center}
由$ f''(x) $在$[a, b]$上连续,根据连续函数的性质可知,在$[a, b]$中存在一点$\eta$,使
\begin{equation*}
\frac{1}{n} \sum_{k=0}^{n-1} f''(\eta_k) = f''(\eta)
\end{equation*}
}

\frame{
\begin{center}
$\Downarrow$
\end{center}
因此,得到复化梯形求积公式的误差估计为
\begin{equation*}
\int_a^b f(x) dx - T_n = -\frac{nh^3}{12} f''(\eta) = -\frac{b-a}{12} h^2 f''(\eta)
\end{equation*}
\begin{center}
$\Downarrow$
\end{center}
当使用复化梯形求积公式求积分时,有时可以使用他们的误差估计式判断$n$应取多大才能满足给定的精度要求。
\begin{center}
$\Downarrow$
\end{center}
就是把积分区间$[a, b]$划分多少等分才能满足给定的积分精度要求。
}



\frame{
\begin{block}{定理}
积分近似值$T_n$与$T_{2n}$有如下关系:
\begin{equation*}
 T_{2n} = \frac{1}{2} \left[ T_n + h \sum_{k=1}^n f \left( a + (2k-1) \frac{b-a}{2n} \right) \right]
\end{equation*}
其中$h = \frac{b-a}{n}$
\end{block}
定理证明:\\ 
把积分区间$[a, b]$进行$2n$等分,
\begin{center}
$\Downarrow$
\end{center}
实际上是在将区间$[a, b]$进行$n$等分的基础之上,
对每个小区间$[x_k, x_{k+1}]$再平分一次,
亦即在每个小区间$[x_k, x_{k+1}]$上插入一个中点。
\begin{center}
$\Downarrow$
\end{center}
}

\frame{
记该中点为$x_{k + \frac{1}{2}}$,则$x_{k + \frac{1}{2}} = x_k + \frac{h}{2} = a + \left( k + \frac{1}{2} \right) h$,
其中$k = 0$, $1$, $2$, $\cdots$, $n-1$
\begin{center}
$\Downarrow$
\end{center}
从而有
\begin{equation*} 
\begin{array}{l l}
\int_a^b f(x) d x & = \sum\limits_{k=0}^{n-1} \int_{x_k}^{x_{k+1}} f(x) dx = \sum\limits_{k=0}^{n-1} \left[ \int_{x_k}^{x_{k+\frac{1}{2}}} f(x) dx + \int_{x_{k + \frac{1}{2}}}^{x_{k+1}} f(x) dx \right] \\
& \\
& \approx  \sum\limits_{k=0}^{n-1} \left\{ \frac{x_{k+\frac{1}{2}} - x_k}{2} \left[ f(x_k) + f(x_{k+\frac{1}{2}})\right] + \frac{x_{k+1} -x_{k+ \frac{1}{2}} }{2} \left[  f(x_{k+\frac{1}{2}}) + f(x_{k+1})\right]\right\} \\
& \\
& = \frac{h}{4} \sum\limits_{k=0}^{n-1} \left[ f(x_k) + 2 f(x_{k + \frac{1}{2}}) + f(x_{k+1})\right] \\
& \\
& = \frac{1}{2} \left[ T_n + h \sum\limits_{k=0}^{n-1} f(x_{k + \frac{1}{2}}) \right]\\
\end{array}
\end{equation*}
}

\frame{
\begin{center}
$\Downarrow$
\end{center}
\begin{equation*} 
\begin{array}{l l}
& = \frac{1}{2} \left[ T_n + h \sum\limits_{k=0}^{n-1} f\left(a + \left( k - 1 + \frac{1}{2} \right) h \right) \right] \\
& \\
& = \frac{1}{2} \left[ T_n + h \sum\limits_{k=0}^{n-1} f\left(a + \left( 2k - 1 \right)  \frac{b-a}{2n} \right) \right] \\
\end{array}
\end{equation*}
\begin{center}
$\Downarrow$
\end{center}
从而可得
\begin{block}{}
\begin{equation*}
T_{2n} =  \frac{1}{2} \left[ T_n + h \sum\limits_{k=1}^{n} f\left( a + \left( 2k - 1 \right)  \frac{b-a}{2n} \right) \right]
\end{equation*}
\end{block}
}

\frame{
\frametitle{复化Simpson求积公式}
把积分区间$[a, b]$进行$n$等分,$n$为偶数,不妨设$n = 2m$ ($m = 1$, $2$, $\cdots$),
则分点为$x_0$, $x_1$, $\cdots$, $x_k$, $\cdots$, $x_{2m}$,
其中$x_k$ $=$ $a$ $+$ $k h$;$k$ $=$ $0$, $1$, $\cdots$, $2m$;$h = (b-a) \slash n$。 \\
使用这些分点把区间$[a, b]$分为$m$个小区间$[x_0, x_2]$, $[x_2, x_4]$, $\cdots$, $[x_{2(m-1)}$, $x_{2m}]$。
根据定积分的性质和求积公式可得:
\begin{equation*}
\begin{array}{l l}
\int_{x_{2(k-1)}}^{x_{2k}} f(x) dx & = \int_{x_{2k-2}}^{x_{2k}} f(x) dx \\
& \\
& \approx \frac{x_{2k} -x_{2k-2}}{6} \left[ f(x_{2k-2}) + 4 f(x_{2k-1}) + f(x_{2k}) \right] \\
& \\
& = \frac{h}{3} \left[ f(x_{2k-2}) + 4 f(x_{2k-1}) + f(x_{2k}) \right]\\
\end{array}
\end{equation*}
}


\frame{
从而有
\begin{equation*}
\begin{array}{l l}
\int_a^b f(x) dx & = \int_{x_0}^{x_2} f(x) dx + \int_{x_2}^{x_4} f(x) dx + \cdots + \int_{x_{2 (m-1)}}^{x_{2m}} f(x) dx \\
& \\
& = \sum\limits_{k=1}^m \int_{x_{2(k-1)}}^{x_{2k}} f(x) dx \\
& \\
& \approx \sum\limits_{k=1}^m  \frac{h}{3} \left[ f(x_{2k-2}) + 4 f(x_{2k-1}) + f(x_{2k}) \right]\\
& \\
& = \frac{h}{3} \left[ f(x_0) + \sum\limits_{k=2}^m f(x_{2k-2}) + 4 \sum\limits_{k=2}^m f(x_{2k-1}) + \sum\limits_{k=2}^{m-1} f(x_{2k}) + f(x_{2m}) \right] \\
& \\
& = \frac{h}{3} \left[  f(a) + f(b) + 4 \sum\limits_{k=1}^m f(x_{2k-1}) + 2 \sum\limits_{k=1}^{m-1} f(x_{2k}) \right] \\
\end{array}
\end{equation*}
上式为复化Simpson求积公式或复化抛物线求积公式,记为$S_n$,其中$n$为等分数。
}

\frame{
\begin{block}{定理}
若$f(x) \in C^4 [a, b]$,则对复化Simpson求积公式$S_n$有
\begin{equation*}
\int_a^b f(x) dx -S_n = -\frac{b-a}{2880} (2h)^4 f^{(4)} (\eta)
\end{equation*}
其中$h=\frac{b-a}{n}$,$a \le \eta \le b$
\end{block}
定理证明:\\ 
把积分区间$[a, b]$进行$2n$等分,
实际上是在将区间$[a, b]$进行$n$等分的基础之上,
其中$n=2m$($m=1$, $2$, $\cdots$),
则分点为$x_0$, $x_1$, $\cdots$, $x_k$, $\cdots$, $x_{2m}$
\begin{center}
$\Downarrow$
\end{center}
因此得到$m$个小区间$[x_0, x_2]$, $[x_2, x_4]$, $\cdots$, $[x_{2(m-1)}, x_{2m}]$。
}

\frame{
\begin{center}
$\Downarrow$
\end{center}
在每个小区间$[x_{2(k-1)}, x_{2k}]$($k = 1$, $2$, $\cdots$, $m$)上采用Simpson求积公式的误差估计式可得:
\begin{equation*}
\begin{array}{l}
\int_a^b f(x) dx  - S_n = \\ \\
\hspace{10mm} \sum\limits_{k=1}^m \left\{ \int_{x_{2(k-1)}}^{x_{2k}} f(x) dx - \frac{x_{2k}-x_{2k-2}}{3} \left[ f(x_{2k-2}) + 4 f(x_{2k-1}) + f(x_{2k}) \right] \right\} \\ \\
\hspace{5mm} = -\frac{(2h)^5}{2880} \sum\limits_{k=1}^m f^{(4)} (\eta_k)
\end{array}
\end{equation*}
其中$x_{2k-2} \le \eta_k \le x_{2k}$
\begin{center}
$\Downarrow$
\end{center}
}

\frame{
由于$f''(x)$在$[a, b]$上连续,根据连续函数的性质,在$[a, b]$中存在一点$\eta$
\begin{equation*}
\frac{1}{m} \sum\limits_{k=1}^m f^{(4)} (\eta_k) = f^{(4)} (\eta) 
\end{equation*}
\begin{center}
$\Downarrow$
\end{center}
由此可得到复化Simpson求积公式的误差估计式
\begin{equation*}
\int_a^b f(x) dx -S_n = -\frac{b-a}{2880} (2h)^4 f^{(4)} (\eta)
\end{equation*}
}

\frame{
\frametitle{复化Simpson求积公式的求积分算法}
\begin{block}{}
\begin{itemize}
\item 输入$a$,$b$和$n$,
\vspace{3mm}
\item $h = \frac{b-a}{n}$
\vspace{3mm}
\item $S_1 = 0$,$S_2 = 0$
\vspace{3mm}
\begin{itemize}
\item 对 $k = 1, 2, \cdots, m$做
\vspace{3mm}
\item $S_1 = S_1 + f( a + (2 \ast k - 1) \ast h)$
\vspace{3mm}
\item 对 $k = 1, 2, \cdots, m-1$做
\vspace{3mm}
\item $S_2 = S_2 + f( a + 2 \ast k \ast h)$
\end{itemize}
\vspace{3mm}
\item 计算$S = \frac{h}{3} [f(a) + f(b) + 4 \ast S_1 + 2 \ast S_2]$
\vspace{3mm}
\item 输出$S$
\end{itemize}
\end{block}
}



%%%%%%%%%%%

\subsection{自动选取步长梯形法}

\frame{
在使用复化梯形求积分公式之前必须要选定$n$
\begin{center}
$\Downarrow$
\end{center}
对复化梯形求积公式的误差估计式的两边求绝对值可得:
\begin{equation*}
\left| \int_a^b f(x) dx - T_n  \right|
= \frac{b-a}{12} h^2 | f''(\eta) |
\le \frac{b-a}{12} h^2 \max_{a \le x \le b} |f''(x)|
\end{equation*}
\begin{center}
$\Downarrow$
\end{center}
\begin{itemize}
\item 如果$n$取的太大 $\Rightarrow$ 复化梯形求积公式与积分精确值的误差变得很小$\Rightarrow$ 计算量增大。
\item 如果$n$取的太小 $\Rightarrow$ 复化梯形求积公式与积分精确值的误差变得很大$\Rightarrow$ 精度难以保证。
\end{itemize}
}

\frame{
如何选择$n$使的积分近似值满足精度要求?
\vspace{5mm}
\begin{center}
$\Downarrow$
\end{center}
\vspace{5mm}
\begin{block}{}
{\Huge 为了避免盲目选取$n$,可以使用自动选取步长梯形法。}
\end{block}
}

\frame{
反复利用公式:
\begin{equation*}
 T_{2n} = \frac{1}{2} \left[ T_n + h \sum_{k=1}^n f \left( a + (2k-1) \frac{b-a}{2n} \right) \right]
\end{equation*}
其中$h = \frac{b-a}{n}$
\begin{center}
$\Downarrow$
\end{center}
\begin{equation*}
\int_a^b f(x) dx - T_n = - \frac{b-a}{12} h^2 f'' (\eta_n) \hspace{5mm} (a \le  \eta_n \le b) 
\end{equation*}
和
\begin{equation*}
\int_a^b f(x) dx - T_{2n} = - \frac{b-a}{12} \left( \frac{h}{2} \right)^2 f'' (\eta_{2n})  \hspace{5mm} (a \le  \eta_{n2} \le b)
\end{equation*}
\begin{center}
$\Downarrow$
\end{center}
}

\frame{
\begin{equation*}
 T_{2n} - T_n = - \frac{b-a}{12} \left( \frac{h}{2} \right)^2 \left[ 4 f'' (\eta_{n}) -f'' (\eta_{2n})  \right] 
\end{equation*}
\begin{center}
$\Downarrow$
\end{center}
当$f''(x)$在$[a, b]$上连续,并假设$n$充分大时,$f''(\eta_n) \approx f''(\eta_{2n})$,则有
\begin{equation*}
\frac{1}{3} \left( T_{2n} - T_n \right) \approx \int_a^b f(x) dx - T_{2n}
\end{equation*}
\begin{center}
$\Downarrow$
\end{center}
因此,假如给定一个精度$\varepsilon$,
要求
\begin{equation*}
\left| \int_a^b f(x) dx - T_{2n} \right| < \varepsilon
\end{equation*}
则可以用$| T_{2n} - T_n | < 3\varepsilon $来判断$T_{2n}$是否满足精度要求。
}

\frame{
\frametitle{自动选取步长梯形法的步骤(I)}
\begin{itemize}
\item 先用梯形求积公式计算出积分的第一次近似值
\begin{equation*}
T_1 = \frac{b-a}{2} \left[ f(a) + f(b) \right]
\end{equation*}
\item 把积分区间$[a, b]$进行2等分,并用复化梯形求积分公式求出积分的第二次近似值
\begin{equation*}
\begin{array}{l l}
T_2 & =  \frac{1}{2} \left[ T_1 + (b-a) \sum\limits_{k=1}^1 f\left( a + (2k-1) \frac{b-a}{2}\right) \right] \\
& \\
& = \frac{T_1}{2} + \frac{b-a}{2} f \left( a + \frac{b-a}{2}\right)
\end{array}
\end{equation*}
判断$|T_2 - T_1| < 3 \varepsilon$是否成立:若成立则停止计算,$T_2$作为积分近似值;否则,进行下一步的计算
\end{itemize}
}

\frame{
\frametitle{自动选取步长梯形法的步骤(II)}
\begin{itemize}
\item  把积分区间$[a, b]$进行4等分,并用复化梯形求积分公式求出积分的第三次近似值
\begin{equation*}
\begin{array}{l l}
T_4 & =  \frac{1}{2} \left[ T_2 + \frac{b-a}{2} \sum\limits_{k=1}^2 f\left( a + (2k-1) \frac{b-a}{4}\right) \right] \\
& \\
& = \frac{T_2}{2} + \frac{b-a}{4}  \sum\limits_{k=1}^2 f \left( a + (2k-1) \frac{b-a}{4}\right)
\end{array}
\end{equation*}
判断$|T_4 - T_2| < 3 \varepsilon$是否成立:若成立则停止计算,$T_4$作为积分近似值;否则,进行下一步的计算
\end{itemize}
}

\frame{
\frametitle{自动选取步长梯形法的步骤(III)}
\begin{itemize}
\item  把积分区间$[a, b]$进行8等分,并用复化梯形求积分公式求出积分的第四次近似值
\begin{equation*}
\begin{array}{l l}
T_8 & =  \frac{1}{2} \left[ T_4 + \frac{b-a}{4} \sum\limits_{k=1}^4 f\left( a + (2k-1) \frac{b-a}{8}\right) \right] \\
& \\
& = \frac{T_4}{2} + \frac{b-a}{8}  \sum\limits_{k=1}^4 f \left( a + (2k-1) \frac{b-a}{8}\right)
\end{array}
\end{equation*}
判断$|T_8 - T_4| < 3 \varepsilon$是否成立:若成立则停止计算,$T_8$作为积分近似值;否则,进行下一步的计算
\item $\cdots$
\item 这样一直迭代下去,直到$|T_{2n} - T_n| < 3 \varepsilon$成立为止,
即$T_{2n}$满足精度 $\varepsilon$要求的积分近似值。
\end{itemize}
}

\frame{
\frametitle{自动选取步长梯形法算法}
\begin{block}{}
\begin{itemize}
\item $S_1$ : 输入$a$,$b$和$\varepsilon$
\vspace{3mm}
\item $S_2$ : $h = \frac{b-a}{2}$, $T_1 = (f(a) + f(b)) \times h$, $n=1$
\vspace{3mm}
\item $S_3$ : 计算$T_0 = T_1$,$S = 0$($T_0$表示前次积分近似值,$T_1$表示后次积分近似值)
\vspace{3mm}
\item $S_4$ : 对 $k = 1, 2, \cdots, n$,计算
\begin{equation*}
S = S + f \left( a + (2k-1) \times \frac{h}{n} \right)
\end{equation*}
\item $S_5$ : $T-1 = \frac{T_0}{2} + S \times \frac{h}{n}$
\vspace{3mm}
\item $S_6$ : 若$|T_1 - T_0| < \varepsilon$,输出$T_1$的值,计算结束。否则,$n = 2 n$,返回$S_3$
\end{itemize}
\end{block}
}



%%%%%%%%%%%

\subsection{Richardson外推算法}

\frame{
假设有一个量$F^{\ast}$,可以是函数,微分,积分等。
\begin{center}
$\Downarrow$
\end{center}
现用一个步长$h$为变量的函数$F_0(h)$来近似代替,且$F^{\ast}$与$F_0(h)$无关,
且给定$F^{\ast}$与$F_0(h)$的误差估计为:
\begin{equation*}
F^{\ast} - F_0(h) = a_1 h^{p_1} + a_2 h^{p_2} + \cdots + a_k h^{p_k} + \cdots
\end{equation*}
其中$0 < p_1 < p_2 < \cdots < p_k < \cdots$;$a_k \neq 0$ ($k = 1, 2, \cdots$)是与$h$无关的常数。
\begin{center}
$\Downarrow$
\end{center}
当$h$适当小时,第一项$ a_1 h^{p_1}$的绝对值远比后面的项的绝对值大。
而第一项之中的$h^{p_1}$是最重要的,它对误差的影响最大。
%$\Rightarrow$
通常称为$F^{\ast}$与$F_0(h)$的误差的阶。
}

\frame{
\begin{center}
$\Downarrow$
\end{center}
当$h$适当小时,误差的阶$h^{p_1}$中的幂$p_1$越大,$F^{\ast}$与$F_0(h)$的误差的绝对值就会越小。
\begin{center}
$\Downarrow$
\end{center}
从函数$F_0(h)$出发构造一个新的函数近似替代$F^{\ast}$,使$F^{\ast}$与这个函数的误差阶的幂比$F^{\ast}$与$F_0(h)$的误差阶的幂更高一些。
\begin{center}
$\Downarrow$
\end{center}
把上式$F^{\ast} - F_0(h) $中的$h$用$qh$代替,可得
\begin{equation*}
F^{\ast} - F_0(qh) = a_1 (qh)^{p_1} + a_2 (qh)^{p_2} + \cdots + a_k (qh)^{p_k} + \cdots
\end{equation*}
其中$q$为常数,并满足$1 - q^{p_1} \neq 0$
\begin{center}
$\Downarrow$
\end{center}
}

\frame{
用$q^{p_1}$乘以$F^{\ast} - F_0(h)$式两边可得
\begin{equation*}
q^{p_1} F^{\ast} - q^{p_1} F_0(h) = a_1 (qh)^{p_1} + a_2 q^{p_1} (qh)^{p_2} + \cdots + a_k q^{p_1} (qh)^{p_k} + \cdots
\end{equation*}
\begin{center}
$\Downarrow$
\end{center}
用$F^{\ast} - F_0(qh)$减去 $q^{p_1} F^{\ast} - q^{p_1} F_0(h)$并整理后可得:
\begin{equation*}
\begin{array}{l}
(1 -q^{p_1})F^{\ast} - \left[ F_0 (qh) - q^{p_1} F_0(h) \right]  \\
= a_2 (q^{p_2} - q^{p_1}) h^{p_2} + a_3 (q^{p_3} - q^{p_1}) h^{p_3} + \cdots + a_k (q^{p_k} - q^{p_1}) h^{p_k} + \cdots
\end{array}
\end{equation*}
\begin{center}
$\Downarrow$
\end{center}
用$1-q^{p_1}$除以上式的两边可得:
\begin{equation*}
\begin{array}{l}
F^{\ast} - \frac{ F_0 (qh) - q^{p_1} F_0(h) }{1 -q^{p_1}}  \\
= \frac{a_2 ( q^{p_2} -q^{p_1} ) h^{p_2}}{1 -q^{p_1}} + \frac{a_3 (q^{p_3} - q^{p_1}) h^{p_3}}{1 -q^{p_1}} + \cdots + \frac{a_k (q^{p_k} - q^{p_1}) h^{p_k}}{1 -q^{p_1}} + \cdots
\end{array}
\end{equation*}
}

\frame{
\begin{center}
$\Downarrow$
\end{center}
若记$F_1(h) = \frac{ F_0 (qh) - q^{p_1} F_0(h) }{1 -q^{p_1}}$和
$a_k^{(1)} = \frac{a_k (q^{p_k} - q^{p_1}) h^{p_k}}{1 -q^{p_1}}$,
$k = 2, 3, \cdots$
则上式可转换为:
\begin{equation*}
F^{\ast} - F_1(h) = a_2^{(1)} h^{p_2} + a_3^{(1)} h^{p_3} + \cdots + a_k^{(1)} h^{p_k} + \cdots
\end{equation*}
其中$a_k^{(1)}$与$h$无关的常数。
\begin{center}
$\Downarrow$
\end{center}
若用函数$F_1(h)$近似代替$F^{\ast}$,则$F_1(h)$与$F^{\ast}$误差的阶为$h^{p_2}$。
\begin{center}
$\Downarrow$
\end{center}
因为$p_1 < p_2$,因此只要$h$适当小,$F_1(h)$与$F^{\ast}$的误差的绝对值比$F_0(h)$与$F^{\ast}$的误差绝对值要小。
}

\frame{
有了$F_1(h)$之后,再从这个函数出发,构造出一个新函数$F_2(h)$近似代替$F^{\ast}$,使得$F^{\ast}$与这个函数的误差的阶的幂比$F^{\ast}$与函数$F_1(h)$的误差的阶的幂更高一些,具体如上述做法类似:
\begin{equation*}
F^{\ast} - F_2(h) = a_3^{(2)} h^{p_3} + a_4^{(2)} h^{p_4} + \cdots + a_k^{(2)} h^{p_k} + \cdots
\end{equation*}
其中 $F_2(h) = \frac{ F_1 (qh) - q^{p_2} F_1(h) }{1 -q^{p_2}}$和
$a_k^{(2)} = \frac{a_k (q^{p_k} - q^{p_2}) h^{p_k}}{1 -q^{p_2}}$,
$k = 3, 4, \cdots$是与$h$无关的常数。
\begin{center}
$\Downarrow$
\end{center}
若用函数$F_2(h)$近似代替$F^{\ast}$,则$F_2(h)$与$F^{\ast}$误差的阶为$h^{p_3}$。
\begin{center}
$\Downarrow$
\end{center}
因为$p_2 < p_3$,因此只要$h$适当小,$F_2(h)$与$F^{\ast}$的误差的绝对值比$F_1(h)$与$F^{\ast}$的误差绝对值要小。
}

\frame{
依次类推,就可以构造出一系列的函数$F_0(h)$,$F_1(h)$,$F_2(h)$,$\cdots$。
\begin{center}
$\Downarrow$
\end{center}
当$h$适当小时,$F^{\ast}$与这些函数的误差的绝对值一个比一个小。
函数的基本形式:
\begin{equation*}
F_m (h) = \frac{F_{m-1}(qh) - q^{p_m} F_{m-1}(h)}{1-q^{p_m}} \hspace{5mm} (m = 1, 2, \cdots )
\end{equation*}
其中$q$为常数,并满足$1-q^{p_m} \neq 0$
\begin{center}
$\Downarrow$
\end{center}
Richardson外推算法,可以用归纳法证明:
\begin{equation*}
F^{\ast} - F_m(h)  = a_{m+1}^{(m)} h^{p_{m+1}} + a_{m+2}^{(m)} h^{p_{m+2}} + \cdots + a_k^{(m)} h^{p_k} + \cdots
\end{equation*}
其中 
$a_k^{(m)} = \frac{a_k^{(m-1)} (q^{p_k} - q^{p_m}) }{1 -q^{p_m}}$,
$k = m+1, m+2, \cdots$是与$h$无关的常数。
}


